
\section{Códigos Fonte do Analisador Léxico}
A seguir será apresentados a implementação do analisador léxico.

\clearpage

\subsection{Leitura dos Tokens com a Ferramenta Flex}

\lstset{
  language=C++,
  basicstyle=\ttfamily\scriptsize, 
  keywordstyle=\color{blue}, 
  stringstyle=\color{verde}, 
  commentstyle=\color{red}, 
  extendedchars=true, 
  showspaces=false, 
  showstringspaces=false, 
  numbers=left,
  numberstyle=\tiny,
  breaklines=true, 
  backgroundcolor=\color{green!10},
  breakautoindent=true, 
  captionpos=b,
  xleftmargin=0pt,
}



\begin{lstlisting}
%{//Primero Campo: Inclusao de Bibliotecas e definicao de constantes

#include "common.hpp"
#include "arvore_sintatica.hpp"
#include "parser.tab.h"

char buffer[30];
extern YYSTYPE yylval;


using namespace std;

string text_id;

%}

%x MULT_LINE_COMMENT_STATE
%x ERROR_DETECTED

DIGITO [0-9]
LETRA [a-zA-Z]

%%


<INITIAL>"//".*"\n"		linhas++;
<INITIAL>"/*"  			{BEGIN(MULT_LINE_COMMENT_STATE);}
<INITIAL>\n				linhas++;
<INITIAL>\r\n			linhas++;
<INITIAL>\t|" "
<INITIAL>if             return IF;
<INITIAL>else			return ELSE;
<INITIAL>while			return WHILE;
<INITIAL>return			return RETURN;
<INITIAL>int			return INT;
<INITIAL>void			return VOID;
<INITIAL>"+"			return SOMA;
<INITIAL>"-"			return SUB;
<INITIAL>"*"			return MULT;
<INITIAL>"/"			return DIV;
<INITIAL>"!""="         return DIFERENTE;
<INITIAL>"=""="			return IGUAL;
<INITIAL>">""="			return MAIOR_IGUAL;
<INITIAL>"<""="			return MENOR_IGUAL;
<INITIAL>"="			return ATRIBUICAO;
<INITIAL>">"			return MAIOR;
<INITIAL>"<"			return MENOR;
<INITIAL>"("			return PARENTESES_ABRE;
<INITIAL>")"			return PARENTESES_FECHA;
<INITIAL>"["			return COLCHETES_ABRE;
<INITIAL>"]"			return COLCHETES_FECHA;
<INITIAL>"{"			return CHAVES_ABRE;
<INITIAL>"}"			return CHAVES_FECHA;
<INITIAL>;              return PONTO_VIRGULA;
<INITIAL>,              return VIRGULA;


<INITIAL>{LETRA}({LETRA}|{DIGITO})*	{ text_id = yytext; return ID;}
<INITIAL>{DIGITO}+					{ return INT_NUM;}

<INITIAL>.                          {BEGIN(ERROR_DETECTED); erro_txt = string(yytext);}



<ERROR_DETECTED>"\n"|"\r\n"        {BEGIN(INITIAL); erro_lexico = true; return ERROR;}
<ERROR_DETECTED>(" "|"\t")         {BEGIN(INITIAL); erro_lexico = true; return ERROR;}
<ERROR_DETECTED>(.)                {erro_txt.append(yytext); }



<MULT_LINE_COMMENT_STATE>"*/"     {BEGIN(INITIAL);}
<MULT_LINE_COMMENT_STATE>(.) ;
<MULT_LINE_COMMENT_STATE>(\n)|("\r\n")  linhas++;
<MULT_LINE_COMMENT_STATE><<EOF>> { perror("Expecting \"*/.\" before end of file \n"); }



%%//Terceiro campo: Rotinas do Usuario

int linhas;

int token;

std::string erro_txt;

FILE *f_in;

void abrirArq(string file_name)
{
  if(f_in = fopen(file_name.c_str(),"r"))
  {
      yyin = f_in;
  }
  else
    perror("lexer");
}

bool erro_lexico=false;

void fecharArq( ){fclose(f_in);}

bool EOFarq( ){return feof(f_in);}

string GetToken( )
{
  string temp;
  string return_token;

  int token=yylex();

  if(token!=0)
  {
    switch(token)
    {
      //Outros
      case INT_NUM: return_token = (string("INT_NUM\t")
                                   + string(yytext) + string("\n"));
      break;
      case FLOAT_NUM: return_token = (string("FLOAT_NUM\t")
                                   + string(yytext) + string("\n"));
      break;
      
      case ID: return_token = (string("ID\t") + string(yytext) + string("\t")
                               + to_string(linhas) + string("\n"));
     break;

      //Palavra Reservadas
      case IF: return_token = (string("IF\n")); break;
      case ELSE: return_token = (string("ELSE\n")); break;
      case WHILE: return_token = (string("WHILE\n")); break;
      case RETURN: return_token = (string("RETURN\n")); break;
      case INT: return_token = (string("INT\n")); break;
      case FLOAT: return_token = (string("FLOAT\n")); break;
      case VOID: return_token = (string("VOID\n")); break;

      //Simbolos Especiais
      case SOMA: return_token = (string("SOMA\n")); break;
      case SUB: return_token = (string("SUB\n")); break;
      case MULT: return_token = (string("MULT\n")); break;
      case DIV: return_token = (string("DIV\n")); break;
      case ATRIBUICAO: return_token = (string("ATRIBUICAO\n")); break;
      case IGUAL: return_token = (string("IGUAL\n")); break;
      case MAIOR: return_token = (string("MAIOR\n")); break;
      case MENOR: return_token = (string("MENOR\n")); break;
      case MAIOR_IGUAL: return_token = (string("MAIOR_IGUAL\n")); break;
      case MENOR_IGUAL: return_token = (string("MENOR_IGUAL\n")); break;
      case PARENTESES_ABRE: return_token = (string("PARENTESES_ABRE\n")); break;
      case PARENTESES_FECHA: return_token = (string("PARENTESES_FECHA\n")); break;
      case COLCHETES_ABRE: return_token = (string("COLCHETES_ABRE\n")); break;
      case COLCHETES_FECHA: return_token = (string("COLCHETES_FECHA\n")); break;
      case CHAVES_ABRE: return_token = (string("CHAVES_ABRE\n")); break;
      case CHAVES_FECHA: return_token = (string("CHAVES_FECHA\n")); break;
      case PONTO_VIRGULA: return_token = (string("PONTO_VIRGULA\n")); break;
      case VIRGULA: return_token = (string("VIRGULA\n")); break;

      case ERROR: return_token = ("ERROR\t" +erro_txt+ "\t" + to_string(linhas) +"\n"); break;

    }

    return return_token;
  }
  else
    return "";
}
\end{lstlisting}
\clearpage