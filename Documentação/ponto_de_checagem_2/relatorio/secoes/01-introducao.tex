 \stepcounter{page}
\setcounter{secnumdepth}{2}

\section{Introduç\~ao}

\subsection{Contextualização}

A Unidade Central de Processamento, ou CPU, é um dos componentes mais vitais do computador, e também um dos que evoluiu de maneira mais rápida.

Tem a função de executar um conjunto de instruções lógicas definidas pelo software.

Para aumentar a produtividade de um programador, e permitir que programas mais complexos sejam produzidos, faz-se o uso de Compiladores.

Um compilador tem a funç\~ao de traduzir uma linguagem para outra, e pode ser utilizado para traduzir uma linguagem de mais alto nivel para o conjunto de instruç\~oes de um determinado processador.

Para facilitar a interação entre o usuário e o sistema computação, existe um programa denominado Sistema Operacional.

O Sistema Operacional tem três principais objetivos: Executar programas do usuário e solucionar seus problemas; Tornar o uso do Sistema Computacional Conveniente; E utilizar o hardware do computador de maneira eficiente.

\subsection{Objetivos}

\subsubsection{Objetivos Gerais}

Este projeto tem como objetivo a implementaç\~ao de um 
sistema operacional capaz de gerenciar processos, memoria e unidades de entrada e saída.

O sistema operacional será desenvolvido na linhagem de programação C-, fazendo o uso de um compilador para traduzi-lo para o conjunto de instruções do sistema computacional onde este sistema será executado.

O sistema computacional foi mapeado em uma FPGA (Field
Programmable Gate Array) Cyclone IV EP4CE115F29C7.

\subsubsection{Objetivos Especificos}

O projeto foi dividido em diversas etapas menores listadas abaixo.

\begin{enumerate}
	\item Ajustes no Compilador e projeto de algoritmos;

	\item Definição do Sistema Operacional a ser projetado, técnicas e algoritmos a serem virtualizados;
	
	\item Adaptações necessárias na plataforma de hardware e implementação da Bios, HD e Sistema de Comunicação entre Componentes;
	
	\item Finalização da implementação do Sistema Operacional.
	
\end{enumerate} 

Atualmente o projeto se encontra em sua segunda etapa.

\clearpage