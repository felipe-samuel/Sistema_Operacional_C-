\section{O Compilador}

O compilador foi implementado fazendo uso da linguagem de programação C++, em conjunto com as ferramentas Flex (para realizar a analise léxica) e Bison (para realizar a análise sintática).

Seu objetivo consiste em traduzir um código escrito na linguagem C- para o conjunto de instruções do Sistema Computacional, listando os erros léxicos, sintáticos e semânticos, caso encontrados.

Nesta seção há uma breve descrição de seu funcionamento, o código fonte deste projeto pode ser encontrado em: 

\url{https://github.com/felipe-samuel/compilador-c-}
 

\subsection{A linguagem C-}

A linguagem de programação C- trata-se de uma simplificação da linguagem C, possuindo as seguintes limitações:

\begin{itemize}
	\item Não podem haver declarações de protótipos de funções;
	
	\item Apenas um tipo do dados, os inteiros (int);
	
	\item Não há utilização de ponteiros;
	
	\item As variáveis devem ser declaradas no inicio do código ou da função;
	
	\item A entrada de dados é feita pela função input, e a saída pela função output.
\end{itemize} 

\subsection{Tokens}

A Tabela \ref{tb:tokens} mostra a lista de tokens aceitos pela linguagem.

\clearpage

\begin{table}[!htb]
	\centering
	\caption{ Lista de Tokens Reconhecidos}
	\label{tb:tokens}
	
	\begin{tabular}{|c|c|c|c|}
		
		\hline
		\textbf{Texto Reconhecido} & \textbf{Token Retornado}\\
		\hline
		
		
		$if$		  		 &                          IF\\
		\hline
		$else$  	   		 &                          ELSE\\
		\hline
		$while$		     &	                        WHILE\\
		\hline
		$return$			 &	                        RETURN\\
		\hline
		$int$			     &	                        INT\\
		\hline
		$void$			 &		                    VOID\\
		\hline
		$+$				 &		                    SOMA\\
		\hline
		$-$				 &		                    SUB\\
		\hline
		$*$				 &		                    MULT\\
		\hline
		$/$				 &		                    DIV\\
		\hline
		$!=$               &                          DIFERENTE\\
		\hline
		$==$				 &                          IGUAL\\
		\hline
		$>=$			     &                          MAIOR\_IGUAL\\
		\hline
		$<=$				 &                          MENOR\_IGUAL\\
		\hline
		$=$				 &	                        ATRIBUICAO\\
		\hline
		$>$				 &		                    MAIOR\\
		\hline
		$<$				 &		                    MENOR\\
		\hline
		$($				 &		                    PARENTESES\_ABRE\\
		\hline
		$)$				 &		                    PARENTESES\_FECHA\\
		\hline
		$[$				 &		                    COLCHETES\_ABRE\\
		\hline
		$]$				 &		                    COLCHETES\_FECHA\\
		\hline
		$\{$				 &	                        CHAVES\_ABRE\\
		\hline
		$\}$				 &		                    CHAVES\_FECHA\\
		\hline
		$;$				 &		                    PONTO\_VIRGULA\\
		\hline
		$,$                &                          VIRGULA\\
		\hline
		
		texto iniciado por uma letra,                    &   ID\\
		podendo conter letras e números  & \\ 
		\hline
		número inteiro    & 					                               INT\_NUM\\
		\hline
		
		sequencia de caracteres diferente   &                   ERROR\\
		dos citados anteriormente             &       \\
		\hline
		
	\end{tabular}
	
\end{table}



\subsection{Expressões Regulares}

O Algoritmo \ref{al:expressaoregular} mostra as expressões regulares usadas nessa linguagem.

\clearpage

\begin{algorithm}

	\begin{center}
		$digito = [0-9]$ \\
		$letra = [a-zA-Z]$ \\
		$INT\_NUM = (+|-)? digito+$\\
		$ID = letra+ (letra|digito)*$\\
	\end{center}
	
	\caption{Expressões Regulares da Linguagem C-}
	\label{al:expressaoregular}
\end{algorithm}


\subsection{Gramática Livre de Contexto}

A gramática livre de contexto aceita por esta linguagem é mostrada no Algoritmo \ref{al:sintaxec-}.

\begin{algorithm}

\lstset{
  language=C++,
  basicstyle=\ttfamily\footnotesize, 
  keywordstyle= , 
  stringstyle= , 
  commentstyle=,
  extendedchars=true, 
  showspaces=false, 
  showstringspaces=false, 
  numbers=left,
  numberstyle=\tiny,
  breaklines=true, 
  backgroundcolor=\color{branco},
  breakautoindent=true, 
  captionpos=b,
  xleftmargin=0pt,
}

\begin{lstlisting}
programa -> programa declaracao
declaracao -> var-declaracao | fun-declaracao | error
var-declaracao -> tipo-especificador var-decl mult-var-decl PONTO_VIRGULA
mult-var-decl -> VIRGULA var-decl mult-var-decl | /*entrada vazia*
var-decl -> ID | ID COLCHETES_ABRE INT_NUM COLCHETES_FECHA
tipo-especificador -> INT | VOID
fun-declaracao -> tipo-especificador ID PARENTESES_ABRE params PARENTESES_FECHA composto-decl | tipo-especificador error composto-decl
params -> param param-lista | VOID  | /*entrada vazia*/
param-lista -> VIRGULA param param-lista  | /* entrada vazia */
param -> tipo-especificador ID | tipo-especificador ID COLCHETES_ABRE COLCHETES_FECHA
composto-decl -> CHAVES_ABRE local-declaracao statement-lista CHAVES_FECHA
local-declaracao -> local-declaracao var-declaracao | local-declaracao error var-declaracao | /* entrada vazia */
statement-lista -> statement-lista statement | statement-lista error statement | /* entrada vazia */ 
statement -> expressao-decl | composto-decl | selecao-decl | iteracao-decl | retorno-decl 
expressao-decl -> expressao PONTO_VIRGULA | PONTO_VIRGULA
selecao-decl -> IF PARENTESES_ABRE expressao PARENTESES_FECHA statement | IF PARENTESES_ABRE expressao PARENTESES_FECHA statement ELSE statement
iteracao-decl -> WHILE PARENTESES_ABRE expressao PARENTESES_FECHA statement
retorno-decl -> RETURN PONTO_VIRGULA | RETURN expressao PONTO_VIRGULA
expressao -> var ATRIBUICAO expressao | simples-expressao
var -> ID | ID COLCHETES_ABRE expressao COLCHETES_FECHA
simples-expressao -> soma-expressao relacional soma-expressao | soma-expressao
relacional: MAIOR_IGUAL | MAIOR | MENOR | MENOR_IGUAL | IGUAL | DIFERENTE
soma-expressao -> soma-expressao soma termo | SOMA termo | SUB  termo | termo
soma -> SOMA | SUB
termo -> termo mult fator | fator
mult -> MULT | DIV
fator -> PARENTESES_ABRE expressao PARENTESES_FECHA | var | ativacao | INT_NUM
ativacao -> ID PARENTESES_ABRE args PARENTESES_FECHA
args -> arg-lista | /* entrada vazia */ 
arg-lista -> arg-lista VIRGULA expressao | expressao
\end{lstlisting}  

\caption{Gramática Livre de Contexto Aceita pela Lingagem de Programação C-}       
\label{al:sintaxec-}

\end{algorithm} 
\clearpage

\subsection{Retorno de Erros}

O compilador retorna erros léxicos que não se encaixam em nenhum token e erros sintáticos que não são satisfazem a gramática livre de contexto.

Os seguintes erros semânticos também são retornados: 

\begin{itemize}
	\item Atribuição de variável a vetor ou vetor a variável;
	\item Retorno de função como vetor;
	\item Criação de variáveis do tipo void;
	\item Utilização de variáveis não declaradas; 
	\item Declaração de variáveis já declaradas;
	\item Programa sem função main.
\end{itemize}


\clearpage